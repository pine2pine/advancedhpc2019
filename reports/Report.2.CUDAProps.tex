\documentclass[10pt]{article}
\usepackage{geometry}                % See geometry.pdf to learn the layout options. There are lots.
\geometry{letterpaper}                   % ... or a4paper or a5paper or ... 
%\geometry{landscape}                % Activate for for rotated page geometry
%\usepackage[parfill]{parskip}    % Activate to begin paragraphs with an empty line rather than an indent

%%%%%%%%%%%%%%%%%%%%
\newcommand{\hide}[1]{}

\usepackage{natbib}
\usepackage{xcolor}
\usepackage{url}
\usepackage{hyperref}
\usepackage{mathtools}

\hide{
\usepackage{amscd}
\usepackage{amsfonts}
\usepackage{amsmath}
\usepackage{amssymb}
\usepackage{amsthm}
\usepackage{cases}		 
\usepackage{cutwin}
\usepackage{enumerate}
\usepackage{epstopdf}
\usepackage{graphicx}
\usepackage{ifthen}
\usepackage{lipsum}
\usepackage{mathrsfs}	
\usepackage{multimedia}
\usepackage{wrapfig}
}

	 
%\input{/usr/local/LATEX/Lee_newcommands.tex}
\newcommand{\itemlist}[1]{\begin{itemize}#1\end{itemize}}
\newcommand{\enumlist}[1]{\begin{enumerate}#1\end{enumerate}}
\newcommand{\desclist}[1]{\begin{description}#1\end{description}}

\newcommand{\Answer}[1]{\begin{quote}{\color{blue}#1}\end{quote}}
\newcommand{\AND}{\wedge}
\newcommand{\OR}{\vee}
\newcommand{\ra}{\rightarrow}
\newcommand{\lra}{\leftrightarrow}

\title {Labwork 2: REPORT}
\author{Nhu-Tung DOAN}
\date{}

\begin{document}
\maketitle

\section{Output from labwork #2}

The following output is obtained by running labwork #2 on ICT5 server:

\begin{verbatim}
student1@ictserver5:/storage/student1/advancedhpc2019/labwork/build$ ./labwork 2
USTH ICT Master 2018, Advanced Programming for HPC.
Warming up...
Starting labwork 2
Number total of GPU : 8

GPU #0:
===========
      Identifier: Tesla K80
      Clock frequency (kHz): 823500
      Number of multiprocessors: 13
      Warp size: 32
      Memory clock frequency (kHz): 2505000
      Memory Bus Width (bits): 384
      Peak Memory Bandwidth (GB/s): 240.480000

GPU #1:
===========
      Identifier: Tesla K80
      Clock frequency (kHz): 823500
      Number of multiprocessors: 13
      Warp size: 32
      Memory clock frequency (kHz): 2505000
      Memory Bus Width (bits): 384
      Peak Memory Bandwidth (GB/s): 240.480000

GPU #2:
===========
      Identifier: Tesla K80
      Clock frequency (kHz): 823500
      Number of multiprocessors: 13
      Warp size: 32
      Memory clock frequency (kHz): 2505000
      Memory Bus Width (bits): 384
      Peak Memory Bandwidth (GB/s): 240.480000

GPU #3:
===========
      Identifier: Tesla K80
      Clock frequency (kHz): 823500
      Number of multiprocessors: 13
      Warp size: 32
      Memory clock frequency (kHz): 2505000
      Memory Bus Width (bits): 384
      Peak Memory Bandwidth (GB/s): 240.480000

GPU #4:
===========
      Identifier: Tesla K80
      Clock frequency (kHz): 823500
      Number of multiprocessors: 13
      Warp size: 32
      Memory clock frequency (kHz): 2505000
      Memory Bus Width (bits): 384
      Peak Memory Bandwidth (GB/s): 240.480000

GPU #5:
===========
      Identifier: Tesla K80
      Clock frequency (kHz): 823500
      Number of multiprocessors: 13
      Warp size: 32
      Memory clock frequency (kHz): 2505000
      Memory Bus Width (bits): 384
      Peak Memory Bandwidth (GB/s): 240.480000

GPU #6:
===========
      Identifier: Tesla K80
      Clock frequency (kHz): 823500
      Number of multiprocessors: 13
      Warp size: 32
      Memory clock frequency (kHz): 2505000
      Memory Bus Width (bits): 384
      Peak Memory Bandwidth (GB/s): 240.480000

GPU #7:
===========
      Identifier: Tesla K80
      Clock frequency (kHz): 823500
      Number of multiprocessors: 13
      Warp size: 32
      Memory clock frequency (kHz): 2505000
      Memory Bus Width (bits): 384
      Peak Memory Bandwidth (GB/s): 240.480000

labwork 2 ellapsed 12.8ms
\end{verbatim}

\subsection*{Thank You! :-)}
\end{document}  
